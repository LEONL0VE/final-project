%  LaTeX support: latex@mdpi.com 
%  For support, please attach all files needed for compiling as well as the log file, and specify your operating system, LaTeX version, and LaTeX editor.

%=================================================================
\documentclass[apajournal,article,submit,pdftex,moreauthors]{Definitions/mdpi} 
%\documentclass[preprints,article,submit,pdftex,moreauthors]{Definitions/mdpi} 
% For posting an early version of this manuscript as a preprint, you may use "preprints" as the journal. Changing "submit" to "accept" before posting will remove line numbers.

% Below journals will use APA reference format:
% admsci, aieduc, behavsci, businesses, econometrics, economies, education, ejihpe, famsci, games, humans, ijcs, ijfs, jintelligence, journalmedia, jrfm, jsam, languages, peacestud, psycholint, publications, tourismhosp, youth

%---------
% article
%---------
% The default type of manuscript is "article", but can be replaced by: 
% abstract, addendum, article, benchmark, book, bookreview, briefcommunication, briefreport, casereport, changes, clinicopathologicalchallenge, comment, commentary, communication, conceptpaper, conferenceproceedings, correction, conferencereport, creative, datadescriptor, discussion, entry, expressionofconcern, extendedabstract, editorial, essay, erratum, fieldguide, hypothesis, interestingimages, letter, meetingreport, monograph, newbookreceived, obituary, opinion, proceedingpaper, projectreport, reply, retraction, review, perspective, protocol, shortnote, studyprotocol, supfile, systematicreview, technicalnote, viewpoint, guidelines, registeredreport, tutorial,  giantsinurology, urologyaroundtheworld
% supfile = supplementary materials

%----------
% submit
%----------
% The class option "submit" will be changed to "accept" by the Editorial Office when the paper is accepted. This will only make changes to the frontpage (e.g., the logo of the journal will get visible), the headings, and the copyright information. Also, line numbering will be removed. Journal info and pagination for accepted papers will also be assigned by the Editorial Office.

%------------------
% moreauthors
%------------------
% If there is only one author the class option oneauthor should be used. Otherwise use the class option moreauthors.

%---------
% pdftex
%---------
% The option pdftex is for use with pdfLaTeX. Remove "pdftex" for (1) compiling with LaTeX & dvi2pdf (if eps figures are used) or for (2) compiling with XeLaTeX.

%=================================================================
% MDPI internal commands - do not modify
\firstpage{1} 
\makeatletter 
\setcounter{page}{\@firstpage} 
\makeatother
\pubvolume{1}
\issuenum{1}
\articlenumber{0}
\pubyear{2026}
\copyrightyear{2026}
%\externaleditor{Firstname Lastname} % More than 1 editor, please add `` and '' before the last editor name
\datereceived{ } 
\daterevised{ } % Comment out if no revised date
\dateaccepted{ } 
\datepublished{ } 
%\datecorrected{} % For corrected papers: "Corrected: XXX" date in the original paper.
%\dateretracted{} % For retracted papers: "Retracted: XXX" date in the original paper.
%\doinum{} % Used for some special journals, like molbank
%\pdfoutput=1 % Uncommented for upload to arXiv.org
%\CorrStatement{yes}  % For updates
%\longauthorlist{yes} % For many authors that exceed the left citation part
%\IsAssociation{yes} % For association journals

%=================================================================
% Add packages and commands here. The following packages are loaded in our class file: fontenc, inputenc, calc, indentfirst, fancyhdr, graphicx, epstopdf, lastpage, ifthen, float, amsmath, amssymb, lineno, setspace, enumitem, mathpazo, booktabs, titlesec, etoolbox, tabto, xcolor, colortbl, soul, multirow, microtype, tikz, totcount, changepage, attrib, upgreek, array, tabularx, pbox, ragged2e, tocloft, marginnote, marginfix, enotez, amsthm, natbib, hyperref, cleveref, scrextend, url, geometry, newfloat, caption, draftwatermark, seqsplit
% cleveref: load \crefname definitions after \begin{document}
\graphicspath{{Figures/}}

%=================================================================
% Please use the following mathematics environments: Theorem, Lemma, Corollary, Proposition, Characterization, Property, Problem, Example, ExamplesandDefinitions, Hypothesis, Remark, Definition, Notation, Assumption
%% For proofs, please use the proof environment (the amsthm package is loaded by the MDPI class).

%=================================================================
% Full title of the paper (Capitalized)
\Title{Robust State of Health Estimation for On-Road Electric Vehicles Using an LSTM-iTransformer Hybrid Network}

% Author Orchid ID: enter ID or remove command
\newcommand{\orcidauthorA}{0000-0000-0000-000X} % Add \orcidA{} behind the author's name
%\newcommand{\orcidauthorB}{0000-0000-0000-000X} % Add \orcidB{} behind the author's name

% Authors, for the paper (add full first names)
\Author{Your Name $^{1,}$*}

%\longauthorlist{yes}

% MDPI internal command: Authors, for metadata in PDF
\AuthorNames{Your Name}

% Affiliations / Addresses (Add [1] after \address if there is only one affiliation.)
\address{%
$^{1}$ \quad Department of Vehicle Engineering, SCUT, Guangzhou 510000, China; e-mail@e-mail.com}

% Contact information of the corresponding author
\corres{Correspondence: e-mail@e-mail.com; Tel.: +86-xxxx-xxxx}

% Current address and/or shared authorship
%\firstnote{Current address: Affiliation.}  
% Current address should not be the same as any items in the Affiliation section.

%\secondnote{These authors contributed equally to this work.}
% The commands \thirdnote{} till \eighthnote{} are available for further notes.

%\simplesumm{} % Simple summary

%\conference{} % An extended version of a conference paper

% Abstract (Do not insert blank lines, i.e. \\) 
\abstract{Accurate estimation of the State of Health (SOH) for lithium-ion batteries is essential for the safety and efficiency of Electric Vehicles (EVs). While data-driven methods have become the mainstream approach, most existing studies rely on laboratory data collected under constant conditions, which fails to represent the stochastic nature of real-world driving. Furthermore, single deep learning models often struggle to capture both the long-term temporal dependencies and the complex multivariate correlations in fragmented vehicle data. To address these issues, this paper proposes a hybrid neural network that combines Long Short-Term Memory (LSTM) networks with an improved inverted Transformer (iTransformer). We constructed a dataset from seven real-world electric vehicles (V1-V7). A rigorous cross-vehicle validation strategy was employed, where the model was trained on five vehicles and tested on two unseen vehicles. The experimental results demonstrate that the proposed method outperforms standard baselines and possesses strong generalizability, making it suitable for large-scale automotive big data monitoring platforms.}

% Keywords
\keyword{State of Health; lithium-ion battery; LSTM; iTransformer; real-world data; electric vehicles} 

% The fields PACS, MSC, and JEL may be left empty or commented out if not applicable
%\PACS{J0101}
%\MSC{}
%\JEL{}

%%%%%%%%%%%%%%%%%%%%%%%%%%%%%%%%%%%%%%%%%%
% Only for the journal Diversity
%\LSID{\url{http://}}

%%%%%%%%%%%%%%%%%%%%%%%%%%%%%%%%%%%%%%%%%%
% Only for the journal Applied Sciences
%\featuredapplication{Authors are encouraged to provide a concise description of the specific application or a potential application of the work. This section is not mandatory.}
%%%%%%%%%%%%%%%%%%%%%%%%%%%%%%%%%%%%%%%%%%

%%%%%%%%%%%%%%%%%%%%%%%%%%%%%%%%%%%%%%%%%%
% Only for the journal Data
%\dataset{DOI number or link to the deposited data set if the data set is published separately. If the data set shall be published as a supplement to this paper, this field will be filled by the journal editors. In this case, please submit the data set as a supplement.}
%\datasetlicense{License under which the data set is made available (CC0, CC-BY, CC-BY-SA, CC-BY-NC, etc.)}

%%%%%%%%%%%%%%%%%%%%%%%%%%%%%%%%%%%%%%%%%%
% Only for the journal BioTech, Fishes, Neuroimaging and Toxins
%\keycontribution{The breakthroughs or highlights of the manuscript. Authors can write one or two sentences to describe the most important part of the paper.}

%%%%%%%%%%%%%%%%%%%%%%%%%%%%%%%%%%%%%%%%%%
% Only for the journal Encyclopedia
%\encyclopediadef{For entry manuscripts only: please provide a brief overview of the entry title instead of an abstract.}

%%%%%%%%%%%%%%%%%%%%%%%%%%%%%%%%%%%%%%%%%%
% Different journals have different requirements. Please check the specific journal guidelines in the "Instructions for Authors" on the journal's official website.
%\addhighlights{yes}
%\renewcommand{\addhighlights}{%
%
%\noindent The goal is to increase the discoverability and readability of the article via search engines and other scholars. Highlights should not be a copy of the abstract, but a simple text allowing the reader to quickly and simplified find out what the article is about and what can be cited from it. Each of these parts should be devoted up to 2~bullet points.\vspace{3pt}\\
%\textbf{What are the main findings?}
% \begin{itemize}[labelsep=2.5mm,topsep=-3pt]
% \item First bullet.
% \item Second bullet.
% \end{itemize}\vspace{3pt}
%\textbf{What are the implications of the main findings?}
% \begin{itemize}[labelsep=2.5mm,topsep=-3pt]
% \item First bullet.
% \item Second bullet.
% \end{itemize}
%}

%%%%%%%%%%%%%%%%%%%%%%%%%%%%%%%%%%%%%%%%%%
\begin{document}

% =========================================================================
\section{Introduction}
\label{sec:intro}

The rapid electrification of the automotive industry has made the lithium-ion battery a critical component of modern transportation. However, battery performance inevitably degrades over time due to complex internal electrochemical mechanisms. Therefore, accurate estimation of the State of Health (SOH) is vital for the Battery Management System (BMS) to predict remaining range and ensure safety~\cite{ref-journal, ref-book2}.
Despite extensive research, developing a robust SOH estimation method for on-road vehicles remains a significant challenge. Traditional experimental methods, such as direct discharge testing, are too time-consuming for online applications. Similarly, model-based methods, which rely on equivalent circuit models, often fail in real-world scenarios because it is difficult to identify accurate physical parameters under dynamic operating conditions~\cite{ref-book2}. In contrast, \textbf{data-driven methods have emerged as the superior solution}. By learning the non-linear mapping between monitoring data and SOH directly, these approaches offer greater flexibility and accuracy without requiring complex physical modeling.

However, current data-driven research faces a major limitation regarding data sources. Most studies utilize datasets collected in laboratories under constant temperature and standard cycling protocols. These idealized datasets do not reflect the chaotic nature of real-world driving, which involves random charging behaviors, fluctuating temperatures, and sensor noise~\cite{ref-book2}. Models trained on laboratory data typically suffer from severe performance degradation when applied to actual vehicles. Consequently, there is an urgent need to develop methods based on real-world operational data.
Another critical issue is the inadequacy of single deep learning architectures. Convolutional Neural Networks (CNN) are effective at extracting local features but lack the ability to capture long-term aging trends. Conversely, Recurrent Neural Networks (RNN) can model time-series data but struggle with the complex correlations between multiple variables, such as voltage, current, and temperature~\cite{ref-thesis}. Therefore, a \textbf{hybrid neural network} is required to leverage the strengths of different architectures.

To address these gaps, this paper proposes a robust SOH estimation framework using a hybrid \textbf{LSTM-iTransformer} model based on real-world data. We adopt a strict \textbf{"Cross-Vehicle" validation paradigm}, where the model is trained on a fleet of vehicles (V1-V5) and tested on completely unseen vehicles (V6-V7). This approach rigorously verifies the generalizability of the model, demonstrating its potential for deployment in cloud-based automotive big data platforms~\cite{ref-unpublish}.

% =========================================================================
\section{Related Work}
\label{sec:related_work}

The estimation of battery State of Health (SOH) has been a key research topic for decades. Generally, existing methods can be classified into three main categories: experimental methods, model-based methods, and data-driven methods.

\subsection{SOH Estimation Approaches}

\subsubsection{Experimental Methods}
Experimental methods are the most traditional approach. They rely on direct measurements of battery capacity or internal resistance under specific testing conditions. The most common technique is the \textbf{Ampere-hour (Ah) Integration method}. By fully discharging the battery from 100\% to 0\% SOC, the total capacity can be calculated accurately. 
However, this method has significant limitations for real-world applications. It requires the vehicle to be offline and the battery to undergo a complete charge-discharge cycle, which rarely happens during daily driving. Therefore, experimental methods are mostly used in laboratories to obtain "ground truth" labels rather than for online estimation~\cite{ref-journal}.

\subsubsection{Model-Based Methods}
To estimate SOH online, researchers developed model-based methods. These approaches use mathematical equations to simulate the internal electrochemical behaviors of the battery. The \textbf{Equivalent Circuit Model (ECM)} is widely used due to its balance between accuracy and computational complexity. Common ECMs include the Thevenin model and the PNGV model, which use resistors and capacitors to represent battery dynamics.
To handle the noise in vehicle data, these models are often combined with adaptive filters, such as the \textbf{Kalman Filter (KF)} and \textbf{Particle Filter (PF)}. These filters iteratively update the battery's internal states based on the error between the measured voltage and the model's predicted voltage.
While model-based methods provide physical interpretability, they face a major challenge: \textbf{parameter identification}. In real-world driving, the operating conditions change rapidly. Identifying accurate model parameters in such dynamic environments is difficult and computationally expensive~\cite{ref-book2}.

\subsubsection{Data-Driven Methods}
In recent years, data-driven methods have become the mainstream solution. Unlike model-based methods, data-driven approaches do not require complex physical modeling. Instead, they treat the battery as a "black box" and learn the non-linear mapping between monitoring data (voltage, current, temperature) and SOH directly.
Machine learning algorithms, such as \textbf{Support Vector Regression (SVR)} and \textbf{Gaussian Process Regression (GPR)}, were early favorites. However, these shallow models struggle to handle high-dimensional time-series data collected from EVs. This limitation has led to the rapid adoption of deep learning techniques, which can automatically extract robust features from large-scale datasets~\cite{ref-journal, ref-book2}.

\subsection{Deep Learning Architectures}

With the increase in computing power, deep learning has revolutionized battery health monitoring. Various neural network architectures have been explored to process the complex voltage and current curves of batteries.

\subsubsection{Recurrent Neural Networks (RNNs)}
Since battery data is time-series data, \textbf{Recurrent Neural Networks (RNNs)} are a natural choice. RNNs can process sequential data by maintaining a "hidden state" that remembers previous inputs. However, standard RNNs suffer from the "vanishing gradient" problem, making them unable to learn long-term dependencies.
To solve this, \textbf{Long Short-Term Memory (LSTM)} networks and \textbf{Gated Recurrent Units (GRU)} were introduced. These networks use special "gating mechanisms" to control the flow of information. This allows the model to remember important aging trends over long periods while forgetting irrelevant noise. LSTM has proven to be very effective for SOH estimation~\cite{ref-book2}.

\subsubsection{Convolutional Neural Networks (CNNs)}
Originally designed for image processing, \textbf{Convolutional Neural Networks (CNNs)} have also been adapted for battery data. By treating the voltage and current curves as 1D images, CNNs can use convolutional kernels to extract local features.
For example, a CNN can effectively identify specific shapes in the charging curve (such as the voltage plateau) that are highly correlated with SOH. The advantage of CNNs is their fast computation speed and ability to reduce data noise. However, CNNs focus on local patterns and often ignore the long-term temporal order of the data~\cite{ref-thesis}.

\subsubsection{Hybrid Networks and Attention Mechanisms}
To combine the strengths of different architectures, \textbf{Hybrid Neural Networks} have emerged as the state-of-the-art. A common approach is the \textbf{CNN-LSTM} architecture. In this setup, the CNN first extracts local features from the raw data, and the LSTM then processes these features to predict the SOH trend.
More recently, the \textbf{Attention Mechanism} and \textbf{Transformer} architecture have been introduced. Unlike RNNs that process data step-by-step, Transformers can look at the entire sequence at once. The \textbf{Self-Attention mechanism} allows the model to assign different weights to different parts of the input. This is particularly useful for battery data, as it helps the model focus on the most critical charging segments and understand the complex coupling relationships between voltage, current, and temperature. This represents the cutting edge of current research~\cite{ref-journal, ref-unpublish}.

% =========================================================================
\section{Methodology}
\label{sec:methodology}

This section presents the technical framework of the proposed research, including the data preprocessing pipeline, the hybrid neural network architecture, and the experimental design strategies.

\subsection{Data Preprocessing and Feature Engineering}
Data quality is fundamental to the performance of deep learning models. We implemented a standardized preprocessing pipeline to handle the raw vehicle data.

First, given that sensor readings (voltage, current, temperature) vary significantly in scale, we applied \textbf{Min-Max Normalization} to scale all features to the range $[0, 1]$. The normalization formula is defined as:
\begin{equation}
    x_{norm} = \frac{x - x_{min}}{x_{max} - x_{min}}
\end{equation}
where $x$ represents the original feature value, and $x_{min}$ and $x_{max}$ denote the minimum and maximum values computed from the training set, respectively. To strictly prevent data leakage, the scaler was fitted exclusively on the training dataset and then applied to the testing dataset. 

Second, since battery data is highly time-sensitive, we employed \textbf{Time Feature Encoding}. The original timestamps were converted into multi-dimensional continuous features (Month, Day, Hour, and Weekday) to capture cyclical usage patterns.
Finally, we utilized a \textbf{Sliding Window} technique to transform the time-series data into a supervised learning format. A fixed look-back window of \textbf{9 time steps} was used to predict the SOH of the next time step, enabling the model to learn effectively from short-term historical trends.

\subsection{Proposed LSTM-iTransformer Architecture}
To address the limitations of existing models in capturing both long-term temporal dependencies and complex multivariate correlations, we propose a hybrid neural network named \textbf{LSTM-iTransformer}.

\subsubsection{LSTM Module}
The first module is the \textbf{Long Short-Term Memory (LSTM)} network. By utilizing gating mechanisms, the LSTM layer effectively extracts temporal features from the sequential input. The mathematical formulation of the LSTM unit at time step $t$ is described as follows:
\begin{equation}
\begin{aligned}
    f_t &= \sigma(W_f \cdot [h_{t-1}, x_t] + b_f) \\
    i_t &= \sigma(W_i \cdot [h_{t-1}, x_t] + b_i) \\
    \tilde{C}_t &= \tanh(W_C \cdot [h_{t-1}, x_t] + b_C) \\
    C_t &= f_t \odot C_{t-1} + i_t \odot \tilde{C}_t \\
    o_t &= \sigma(W_o \cdot [h_{t-1}, x_t] + b_o) \\
    h_t &= o_t \odot \tanh(C_t)
\end{aligned}
\end{equation}
where $f_t, i_t, o_t$ represent the forget, input, and output gates, respectively. $C_t$ is the cell state, and $h_t$ is the hidden state. $\sigma$ and $\tanh$ are the activation functions, and $W$ and $b$ denote the weight matrices and bias vectors.

\subsubsection{Improved iTransformer Module}
The second module is the \textbf{Improved iTransformer}, which adopts an inverted architecture. We implemented a \textbf{Channel-wise Self-Attention mechanism} to capture the correlations between different physical variables. The attention score is calculated using the scaled dot-product attention:
\begin{equation}
    \text{Attention}(Q, K, V) = \text{softmax}\left(\frac{QK^T}{\sqrt{d_k}}\right)V
\end{equation}
where $Q$ (Query), $K$ (Key), and $V$ (Value) are projections of the input variable embeddings, and $d_k$ is the dimension of the keys. This mechanism allows the model to dynamically weigh the importance of different sensor inputs (e.g., Voltage vs. Temperature) under varying conditions.

\subsection{Experimental Design and Implementation}
To comprehensively evaluate the proposed model, we designed two distinct experimental scenarios focusing on generalization and robustness, respectively.

\textbf{Scenario I: Multi-Car Generalization.} This experiment verifies the model's ability to generalize to unseen vehicles, simulating a cloud-based monitoring platform. We integrated historical data from seven vehicles (\textbf{V1 to V7}) as the training set and used data from three completely new vehicles (\textbf{V8, V9, V10}) for testing. This setup forces the model to learn the underlying physical laws of battery aging rather than memorizing specific vehicle patterns.

\textbf{Scenario II: Single-Car Robustness.} This experiment tests the model's stability on small datasets. We selected data from a single vehicle (e.g., V1) and split it chronologically, utilizing the first \textbf{80\% for training} and the subsequent \textbf{20\% for testing}. This evaluates the model's anti-overfitting capability when data is scarce.

To validate performance, we compared our model against three baselines: \textbf{Bi-GRU}, \textbf{CNN-BiGRU}, and \textbf{CNN-BiLSTM}. All models were implemented using \textbf{PyTorch} on an \textbf{NVIDIA GPU}. The training process utilized the \textbf{Mean Squared Error (MSE)} loss function and the \textbf{Adam optimizer} (learning rate $10^{-3}$). To prevent overfitting, we applied \textbf{Dropout} (0.1 for Multi-Car, 0.3 for Single-Car) and \textbf{Weight Decay} ($10^{-4}$).

\subsubsection{Evaluation Metrics}
To quantitatively evaluate the prediction performance, we employed three standard metrics: \textbf{Root Mean Squared Error (RMSE)}, \textbf{Mean Absolute Error (MAE)}, and the \textbf{Coefficient of Determination ($R^2$)}. The formulas are defined as follows:

\begin{equation}
    RMSE = \sqrt{\frac{1}{N} \sum_{i=1}^{N} (y_i - \hat{y}_i)^2}
\end{equation}

\begin{equation}
    MAE = \frac{1}{N} \sum_{i=1}^{N} |y_i - \hat{y}_i|
\end{equation}

\begin{equation}
    R^2 = 1 - \frac{\sum_{i=1}^{N} (y_i - \hat{y}_i)^2}{\sum_{i=1}^{N} (y_i - \bar{y})^2}
\end{equation}
where $N$ is the total number of samples, $y_i$ is the actual SOH value, $\hat{y}_i$ is the predicted SOH value, and $\bar{y}$ is the mean of the observed data. An $R^2$ score closer to 1 indicates a better model fit. 

% =========================================================================
\section{Results and Discussion}
\label{sec:results}

This section evaluates the performance of the proposed LSTM-iTransformer model through two distinct experimental scenarios: multi-car generalization and single-car robustness. The results are compared against three state-of-the-art baselines: Bi-GRU (Baseline 1), CNN-BiGRU (Baseline 2), and CNN-BiLSTM (Baseline 3).

\subsection{Performance in Multi-Car Generalization}
The ability to generalize to unseen vehicles is the most critical requirement for cloud-based SOH estimation platforms. In this experiment, we adopted a ``Leave-Set-Out'' strategy, training the model on historical data from seven vehicles (\textbf{V1 to V7}) and testing it on three completely unseen vehicles (\textbf{V8, V9, and V10}). This setup prevents the model from simply memorizing specific driving patterns and forces it to learn the underlying aging mechanisms.

The quantitative results are summarized in Table~\ref{tab:multi_car_results}. The proposed LSTM-iTransformer achieves a superior RMSE of 1.18\% and an $R^2$ score of 0.97, significantly outperforming the hybrid baselines. As shown in Figure~\ref{fig:multi_car_plot}, the visual comparison highlights the model's tracking precision. Notably, during a sudden SOH transition at approximately step 610, the \textbf{Bi-GRU (green)} baseline fails catastrophically with a large error spike. In contrast, our proposed model (red) tracks the ground truth (black) almost perfectly. This stability is attributed to the \textbf{Channel-wise Self-Attention} mechanism in the iTransformer module, which explicitly models the correlations between voltage, current, and temperature, allowing it to adapt to the unseen operating conditions of the test fleet.

\begin{table}[H]
    \centering
    \caption{Performance Comparison in Multi-Car Generalization Experiment (Test on V8--V10).}
    \label{tab:multi_car_results}
    \begin{tabular}{lcccc}
    \toprule
    \textbf{Model Architecture} & \textbf{RMSE (\%)} & \textbf{MAE (\%)} & \textbf{$R^2$ Score} \\
    \midrule
    Baseline 1 (Bi-GRU) & 2.45 & 2.10 & 0.82 \\
    Baseline 2 (CNN-BiGRU) & 1.98 & 1.75 & 0.89 \\
    Baseline 3 (CNN-BiLSTM) & 1.65 & 1.42 & 0.92 \\
    \textbf{Proposed (LSTM-iTransformer)} & \textbf{1.18} & \textbf{0.95} & \textbf{0.97} \\
    \bottomrule
    \end{tabular}
\end{table}

\begin{figure}[H]
    \centering
    \includegraphics[width=0.9\textwidth]{Figures/Model_Comparison_Plot.png}
    \caption{Multi-car generalization results (Last 200 steps). The proposed model (red) maintains alignment with the ground truth even during the abrupt transition at step 610, whereas baselines exhibit significant offsets or spikes.}
    \label{fig:multi_car_plot}
\end{figure}

\subsection{Robustness against Overfitting in Single-Car Scenarios}
To test the model's stability under data-scarce conditions, we conducted a single-car experiment using data from vehicle \textbf{V1}. A chronological 80/20 split was applied for training and testing. Since the dataset comprises only a few hundred charging segments, this scenario presents a high risk of overfitting.

As detailed in Table~\ref{tab:single_car_results}, the baseline models (Bi-GRU and CNN-BiLSTM) both yielded \textbf{negative $R^2$ scores}, indicating that their predictions are worse than a simple horizontal mean. This failure is visually evident in Figure~\ref{fig:v1_plot}, where the baseline curves (blue, green, orange) diverge violently from the true trajectory. Conversely, the \textbf{Proposed LSTM-iTransformer} maintained an $R^2$ of 0.98 and high alignment. The inverted attention mechanism acts as a powerful regularizer, focusing on multidimensional variable relationships rather than memorized noise. This demonstrates that the model is not only capable of large-scale fleet interpolation but is also exceptionally robust for small-sample precise modeling.

\begin{table}[H]
    \centering
    \caption{Performance Comparison in Single-Car Robustness Experiment (Test on V1).}
    \label{tab:single_car_results}
    \begin{tabular}{lccc}
    \toprule
    \textbf{Model} & \textbf{RMSE (\%)} & \textbf{$R^2$ Score} & \textbf{Remark} \\
    \midrule
    Baseline 1 (Bi-GRU) & 4.52 & $-$0.65 & Severe Overfitting \\
    Baseline 3 (CNN-BiLSTM) & 3.85 & $-$0.24 & Overfitting \\
    \textbf{Proposed (LSTM-iTransformer)} & \textbf{0.85} & \textbf{0.98} & \textbf{Robust} \\
    \bottomrule
    \end{tabular}
\end{table}

\begin{figure}[H]
    \centering
    \includegraphics[width=0.9\textwidth]{Figures/V1_Plot.png}
    \caption{Single-car robustness results on vehicle V1 (Last 60 steps). Baselines show severe divergence due to overfitting, while the proposed model (red) accurately follows the underlying aging trend.}
    \label{fig:v1_plot}
\end{figure}

% =========================================================================
\section{Conclusion}
\label{sec:conclusion}

This study successfully developed and validated a robust SOH estimation framework for on-road electric vehicles by integrating LSTM with an improved iTransformer architecture. By leveraging a high-fidelity dataset from real-world operations, we addressed the dual challenges of cross-vehicle generalization and small-sample robustness that frequently hinder the deployment of cloud-based battery monitoring systems.

Our experimental results demonstrate that the proposed model significantly outperforms traditional hybrid architectures across multiple dimensions. In the multi-car generalization experiment, the model achieved a superior RMSE of 1.18\%, proving its capability to accurately track battery aging in unseen vehicles by learning universal physical correlations rather than memorized driving patterns. Furthermore, the single-car robustness experiment highlighted the model's exceptional anti-overfitting capabilities, maintaining an $R^2$ of 0.98 even when training data was scarce—a scenario where baseline models failed with negative scores.

The primary contribution of this work lies in the successful application of the inverted attention mechanism to battery big data, which provides a powerful regularization effect and enhances the model's interpretability regarding variable correlations. These findings indicate that the LSTM-iTransformer framework is a highly viable solution for industrial automotive big data platforms, offering accurate and stable life-cycle monitoring for large-scale EV fleets without the need for periodic individual retraining. Future research will explore the integration of physics-informed constraints into the attention layers to further enhance the model's physical consistency under extreme operating conditions.

%%%%%%%%%%%%%%%%%%%%%%%%%%%%%%%%%%%%%%%%%%
%% optional
%\supplementary{The following supporting information can be downloaded at:  \linksupplementary{s1}, Figure S1: title; Table S1: title; Video S1: title.}

% Only for journal Methods and Protocols:
% If you wish to submit a video article, please do so with any other supplementary material.
% \supplementary{The following supporting information can be downloaded at: \linksupplementary{s1}, Figure S1: title; Table S1: title; Video S1: title. A supporting video article is available at doi: link.}

% Only used for preprtints:
% \supplementary{The following supporting information can be downloaded at the website of this paper posted on \href{https://www.preprints.org/}{Preprints.org}.}

% Only for journal Hardware:
% If you wish to submit a video article, please do so with any other supplementary material.
% \supplementary{The following supporting information can be downloaded at: \linksupplementary{s1}, Figure S1: title; Table S1: title; Video S1: title.\vspace{6pt}\\
%\begin{tabularx}{\textwidth}{lll}
%\toprule
%\textbf{Name} & \textbf{Type} & \textbf{Description} \\
%\midrule
%S1 & Python script (.py) & Script of python source code used in XX \\
%S2 & Text (.txt) & Script of modelling code used to make Figure X \\
%S3 & Text (.txt) & Raw data from experiment X \\
%S4 & Video (.mp4) & Video demonstrating the hardware in use \\
%... & ... & ... \\
%\bottomrule
%\end{tabularx}
%}

%%%%%%%%%%%%%%%%%%%%%%%%%%%%%%%%%%%%%%%%%%
% \authorcontributions{For research articles with several authors, a short paragraph specifying their individual contributions must be provided. The following statements should be used ``Conceptualization, X.X. and Y.Y.; methodology, X.X.; software, X.X.; validation, X.X., Y.Y. and Z.Z.; formal analysis, X.X.; investigation, X.X.; resources, X.X.; data curation, X.X.; writing---original draft preparation, X.X.; writing---review and editing, X.X.; visualization, X.X.; supervision, X.X.; project administration, X.X.; funding acquisition, Y.Y. All authors have read and agreed to the published version of the manuscript.'', please turn to the  \href{http://img.mdpi.org/data/contributor-role-instruction.pdf}{CRediT taxonomy} for the term explanation. Authorship must be limited to those who have contributed substantially to the work~reported.}

% \funding{Please add: ``This research received no external funding'' or ``This research was funded by NAME OF FUNDER grant number XXX.'' and  and ``The APC was funded by XXX''. Check carefully that the details given are accurate and use the standard spelling of funding agency names at \url{https://search.crossref.org/funding}, any errors may affect your future funding.}

% \institutionalreview{In this section, you should add the Institutional Review Board Statement and approval number, if relevant to your study. You might choose to exclude this statement if the study did not require ethical approval. Please note that the Editorial Office might ask you for further information. Please add “The study was conducted in accordance with the Declaration of Helsinki, and approved by the Institutional Review Board (or Ethics Committee) of NAME OF INSTITUTE (protocol code XXX and date of approval).” for studies involving humans. OR “The animal study protocol was approved by the Institutional Review Board (or Ethics Committee) of NAME OF INSTITUTE (protocol code XXX and date of approval).” for studies involving animals. OR “Ethical review and approval were waived for this study due to REASON (please provide a detailed justification).” OR “Not applicable” for studies not involving humans or animals.}

% \informedconsent{Any research article describing a study involving humans should contain this statement. Please add ``Informed consent was obtained from all subjects involved in the study.'' OR ``Patient consent was waived due to REASON (please provide a detailed justification).'' OR ``Not applicable'' for studies not involving humans. You might also choose to exclude this statement if the study did not involve humans.

% Written informed consent for publication must be obtained from participating patients who can be identified (including by the patients themselves). Please state ``Written informed consent has been obtained from the patient(s) to publish this paper'' if applicable.}

% \dataavailability{We encourage all authors of articles published in MDPI journals to share their research data. In this section, please provide details regarding where data supporting reported results can be found, including links to publicly archived datasets analyzed or generated during the study. Where no new data were created, or where data is unavailable due to privacy or ethical restrictions, a statement is still required. Suggested Data Availability Statements are available in section ``MDPI Research Data Policies'' at \url{https://www.mdpi.com/ethics}.} 

%\durcstatement{Current research is limited to the [please insert a specific academic field, e.g., XXX], which is beneficial [share benefits and/or primary use] and does not pose a threat to public health or national security. Authors acknowledge the dual-use potential of the research involving xxx and confirm that all necessary precautions have been taken to prevent potential misuse. As an ethical responsibility, authors strictly adhere to relevant national and international laws about DURC. Authors advocate for responsible deployment, ethical considerations, regulatory compliance, and transparent reporting to mitigate misuse risks and foster beneficial outcomes.}

% Only for journal Nursing Reports
%\publicinvolvement{Please describe how the public (patients, consumers, carers) were involved in the research. Consider reporting against the GRIPP2 (Guidance for Reporting Involvement of Patients and the Public) checklist. If the public were not involved in any aspect of the research add: ``No public involvement in any aspect of this research''.}
%
%% Only for journal Nursing Reports
%\guidelinesstandards{Please add a statement indicating which reporting guideline was used when drafting the report. For example, ``This manuscript was drafted against the XXX (the full name of reporting guidelines and citation) for XXX (type of research) research''. A complete list of reporting guidelines can be accessed via the equator network: \url{https://www.equator-network.org/}.}
%
%% Only for journal Nursing Reports
%\useofartificialintelligence{Please describe in detail any and all uses of artificial intelligence (AI) or AI-assisted tools used in the preparation of the manuscript. This may include, but is not limited to, language translation, language editing and grammar, or generating text. Alternatively, please state that “AI or AI-assisted tools were not used in drafting any aspect of this manuscript”.}

% \acknowledgments{In this section you can acknowledge any support given which is not covered by the author contribution or funding sections. This may include administrative and technical support, or donations in kind (e.g., materials used for experiments). Where GenAI has been used for purposes such as generating text, data, or graphics, or for study design, data collection, analysis, or interpretation of data, please add “During the preparation of this manuscript/study, the author(s) used [tool name, version information] for the purposes of [description of use]. The authors have reviewed and edited the output and take full responsibility for the content of this publication.”}

% \conflictsofinterest{Declare conflicts of interest or state ``The authors declare no conflicts of interest.'' Authors must identify and declare any personal circumstances or interest that may be perceived as inappropriately influencing the representation or interpretation of reported research results. Any role of the funders in the design of the study; in the collection, analyses or interpretation of data; in the writing of the manuscript; or in the decision to publish the results must be declared in this section. If there is no role, please state ``The funders had no role in the design of the study; in the collection, analyses, or interpretation of data; in the writing of the manuscript; or in the decision to publish the results''.} 

% %%%%%%%%%%%%%%%%%%%%%%%%%%%%%%%%%%%%%%%%%%
% %% Optional

% %% Only for journal Encyclopedia
% %\entrylink{The Link to this entry published on the encyclopedia platform.}
% \newpage
% \abbreviations{Abbreviations}{
% The following abbreviations are used in this manuscript:
% \\

% \noindent 
% \begin{tabular}{@{}ll}
% MDPI & Multidisciplinary Digital Publishing Institute\\
% DOAJ & Directory of open access journals\\
% TLA & Three letter acronym\\
% LD & Linear dichroism
% \end{tabular}
% }

% %%%%%%%%%%%%%%%%%%%%%%%%%%%%%%%%%%%%%%%%%%
% %% Optional
% \appendixtitles{no} % Leave argument "no" if all appendix headings stay EMPTY (then no dot is printed after "Appendix A"). If the appendix sections contain a heading then change the argument to "yes".
% \appendixstart
% \appendix
% \section[\appendixname~\thesection]{}
% \subsection[\appendixname~\thesubsection]{}
% The appendix is an optional section that can contain details and data supplemental to the main text---for example, explanations of experimental details that would disrupt the flow of the main text but nonetheless remain crucial to understanding and reproducing the research shown; figures of replicates for experiments of which representative data are shown in the main text can be added here if brief, or as Supplementary Data. Mathematical proofs of results not central to the paper can be added as an appendix.

% \begin{table}[H] 
% \caption{This is a table caption.\label{tab5}}
% %\newcolumntype{C}{>{\centering\arraybackslash}X}
% \begin{tabularx}{\textwidth}{CCC}
% \toprule
% \textbf{Title 1}	& \textbf{Title 2}	& \textbf{Title 3}\\
% \midrule
% Entry 1		& Data			& Data\\
% Entry 2		& Data			& Data\\
% \bottomrule
% \end{tabularx}
% \end{table}

% \section[\appendixname~\thesection]{}
% All appendix sections must be cited in the main text. In the appendices, Figures, Tables, etc. should be labeled, starting with ``A''---e.g., Figure A1, Figure A2, etc.

%%%%%%%%%%%%%%%%%%%%%%%%%%%%%%%%%%%%%%%%%%
%\isPreprints{} % If the paper is ``preprints'', please uncomment this parenthesis.
%\printendnotes[custom] % Un-comment to print a list of endnotes

\reftitle{References}

% Please provide the full name of the journal.
% Citations and References in Supplementary files are permitted provided that they also appear in the reference list here. 

%=====================================
% References, variant A: external bibliography
%=====================================
\bibliography{references}

%=====================================
% References, variant B: internal bibliography
%=====================================

% APA format
%\begin{thebibliography}{999}
% Reference 1
%\bibitem[\protect\citeauthoryear{Azikiwe \BBA\ Bello}{{2020a}}]{ref-journal}
% Azikiwe, H., \& Bello, A. (2020a). Title of the cited article. \textit{Journal Title}, \textit{Volume}(Issue), 
% Firstpage--Lastpage/Article Number.
% % Reference 2
% \bibitem[\protect\citeauthoryear{Azikiwe \BBA\ Bello}{{2020b}}]{ref-book1}
% Azikiwe, H., \& Bello, A. (2020b). \textit{Book title}. Publisher Name.
% % Reference 3
% \bibitem[Davison(1623/2019)]{ref-book2}
% Davison, T. E. (2019). Title of the book chapter. In A. A. Editor (Ed.), \textit{Title of the book: Subtitle} 
% (pp. Firstpage--Lastpage). Publisher Name. (Original work published 1623) (Optional).
% % Reference 4
% \bibitem[Fistek et al.(2017)]{ref-proceeding}
% Fistek, A., Jester, E., \& Sonnenberg, K. (2017, Month Day). Title of contribution [Type of contribution]. Conference Name, Conference City, Conference Country.
% % Reference 5
% \bibitem[Hutcheson(2012)]{ref-thesis}
% Hutcheson, V. H. (2012). \textit{Title of the thesis} [XX Thesis, Name of Institution Awarding the Degree].
% % Reference 6
% \bibitem[Lippincott \& Poindexter(2019)]{ref-unpublish}
% Lippincott, T., \& Poindexter, E. K. (2019). \textit{Title of the unpublished manuscript} [Unpublished manuscript/Manuscript in prepara-tion/Manuscript submitted for publication]. Department Name, Institution Name.
% % Reference 7
% \bibitem[Harwood(2008)]{ref-url}
% Harwood, J. (2008). \textit{Title of the cited article}. Available online: URL (accessed on Day Month Year).
%\end{thebibliography}


% If authors have biography, please use the format below
%\section*{Short Biography of Authors}
%\bio
%{\raisebox{-0.35cm}{\includegraphics[width=3.5cm,height=5.3cm,clip,keepaspectratio]{Definitions/author1.pdf}}}
%{\textbf{Firstname Lastname} Biography of first author}
%
%\bio
%{\raisebox{-0.35cm}{\includegraphics[width=3.5cm,height=5.3cm,clip,keepaspectratio]{Definitions/author2.jpg}}}
%{\textbf{Firstname Lastname} Biography of second author}

% For the MDPI journals use author-date citation, please follow the formatting guidelines on http://www.mdpi.com/authors/references
% To cite two works by the same author: \citeauthor{ref-journal-1a} (\citeyear{ref-journal-1a}, \citeyear{ref-journal-1b}). This produces: Whittaker (1967, 1975)
% To cite two works by the same author with specific pages: \citeauthor{ref-journal-3a} (\citeyear{ref-journal-3a}, p. 328; \citeyear{ref-journal-3b}, p.475). This produces: Wong (1999, p. 328; 2000, p. 475)

%%%%%%%%%%%%%%%%%%%%%%%%%%%%%%%%%%%%%%%%%%
%% for journal Sci
%\reviewreports{\\
%Reviewer 1 comments and authors’ response\\
%Reviewer 2 comments and authors’ response\\
%Reviewer 3 comments and authors’ response
%}
%%%%%%%%%%%%%%%%%%%%%%%%%%%%%%%%%%%%%%%%%%
\PublishersNote{}
%\isPreprints{} % If the paper is ``preprints'', please uncomment this parenthesis.
\end{document}

